\graphicspath{{chapters/chapter2/}}

\chapter{Background}

\section{The Minimum-Cost Flow Problem}

The minimum-cost flow problem, also known as the transshipment problem, is regarded as the most fundamental flow problem \parencite[4]{AhujaEtAl1993}. The problem is to find the cheapest way to transport a commodity through a network in order to satisfy demands at some nodes from supplies at other nodes. The problem arises in many scenarios such as distributing goods from manufacturers to retailers, the flow of parts through a production line and routing traffic through a road network \parencite[5]{AhujaEtAl1993}.

Formally, the minimum-cost flow problem can be defined as follows. Let $G=(N,E)$ be a directed graph where $N$ is a set of nodes and $E$ is a set of edges represented as ordered pairs of nodes. Edges from nodes to themselves are not allowed. Every node $i \in N$ has an associated supply/demand $d_i$. If $d_i > 0$ node $i$ is a supply node, if $d_i < 0$ node $i$ is a demand node and if $d_i = 0$ node $i$ is known as a transshipment node. It is required that the supplies and demands of all nodes sum to zero, i.e. $\sum_{i \in N} d_i = 0$. In addition, every edge $(i,j) \in E$ has an associated lower capacity $l_{ij}$, upper capacity $u_{ij}$ and cost function $c_{ij}(x)$.

\begin{alignat}{3} 
\label{eq:eq1} & \text{Minimize}   \qquad && \sum_{(i,j) \in E} c_{ij}(x_{ij})\\[1.5ex]
\label{eq:eq2} & \text{subject to} \qquad && \sum_{j \in N^+(i)} x_{ij} - \sum_{j \in N^-(i)} x_{ji} = d_i, \qquad & \forall i &\in N\\[1.5ex]
\label{eq:eq3} &                   \qquad && 0 \leq l_{ij} \leq x_{ij} \leq u_{ij}, \qquad & \forall (i,j) &\in E \\[1.5ex]
               & \text{where}      \qquad && N^+(i) = \{j:(i,j) \in E\}\\[1.5ex]
               &                   \qquad && N^-(i) = \{j:(j,i) \in E\}
\end{alignat}

\autoref{eq:eq2} is known as the flow conservation constraint. It states that the difference in flow leaving and entering a node must be equal to the supply or demand of the node. The first term is the amount of flow leaving node $i$ and the second term is the amount of flow entering node $i$. \autoref{eq:eq3} is known as the capacity constraint and states that the flow through an edge is bounded by the lower and upper capacities of the edge.

\subsection{Concave Cost Functions}

When the cost functions are linear the minimum cost flow problem can be efficiently solved in polynomial time by for instance the network simplex algorithm. Networks with convex cost functions can also be solved in linear time. However, if the cost functions are concave the minimum cost flow problem is NP-hard and thus no polynomial time algorithm exists unless $P=NP$. Concave cost functions arise in many application areas such as production planning, transportation and communication network design, facilities location, and VLSI design due to start-up costs, discounts and economies of scale \parencite[77]{Guisewite1990}.

A function is said to be concave on an interval $X$, if it satisfies the following property:
\begin{equation}
f(tx_1+(1-t)x_2) \geq tf(x_1)+(1-t)f(x_2), \quad \forall x_1,x_2 \in X, \quad \forall t \in [0,1]
\end{equation}
Intuitively, this can be understood in the following way. $tx_1+(1-t)x_2$ is a number between $x_1$ and $x_2$ and $tf(x_1)+(1-t)f(x_2)$ is the parametric equation of the straight line between the points $(x_1,f(x_1))$ and $(x_2,f(x_2))$. The definition can thus intuitively be understood as that the point $(tx_1+(1-t)x_2,f(tx_1+(1-t)x_2))$ lies above the straight line between the points $(x_1,f(x_1))$ and $(x_2,f(x_2))$. This is illustrated in figure \ref{fig:my_label}.

\begin{figure}
    \centering
    \begin{tikzpicture}[domain=0.5:9.5]
        \draw[->] (0,0) -- (10,0) node[right] {$x$};
        \draw[->] (0,0) -- (0,6) node[above] {$y$};
        \draw[-] (1.5,2.55) -- (7.5,3.75);
        \draw[-] (4.5,3.15) -- (4.5,4.95);
        \draw[color=gray] (1.5,0) -- (1.5,6);
        \draw[color=gray] (7.5,0) -- (7.5,6);
        \draw[color=gray] (0,4.95) -- (10,4.95);
        \draw[color=gray] (0,3.15) -- (10,3.15);
        \draw[-] (1.5,-0.1) -- (1.5,0.1) node[below=6pt] {$x_1$};
        \draw[-] (7.5,-0.1) -- (7.5,0.1) node[below=6pt] {$x_2$};
        \draw[-] (4.5,-0.1) -- (4.5,0.1) node[below=6pt] {$tx_1+(1-t)x_2$};
        \draw[-] (-0.1,4.95) -- (0.1,4.95);
        \draw[-] (-0.1,3.15) -- (0.1,3.15);
        \draw[color=red] plot[id=x, samples=100] function{-0.2*((x-5)*(x-5))+5};
        \draw[fill=black] (1.5,2.55) circle (1pt);
        \draw[fill=black] (7.5,3.75) circle (1pt);
        \draw[fill=black] (4.5,4.95) circle (1pt) node[above] {$f(tx_1+(1-t)x_2)$};
        \draw[fill=black] (4.5,3.15) circle (1pt) node[below=6pt] {$tf(x_1)+(1-t)f(x_2)$};
    \end{tikzpicture}
    \caption{A concave function}
    \label{fig:my_label}
\end{figure}

\subsection{Solution Methods}
\emph{Kommentar fra jørgen: Jeg tenkte å drøfte ulike løsningsmetoder i lit. review. Du må gjerne snakke om løsningsmetoder her også, men jeg tenker at det sikkert blir en tekst hvor man refererer mye til annen literatur.}
%end Minimum-Cost Flow Problem


\section{Ant Colony Optimization}

\subsection{General background; (history and development?)}
\subsection{Solution Construction}
\begin{itemize}
    \item[] \textbf{Edge selection} (discuss our specific solution construction method here, keep it abstract)
\end{itemize}
\subsection{Using Pheromones}

Ant colony optimization (ACO) is a constructive and population-based metaheuristic for solving combinatorial optimization problems inspired by how ants use pheromones to find the shortest path between their nest and a food source. It was first defined by \textcite{Dorigo1999a} and later elaborated by \textcite{Dorigo1999,Dorigo2004}.

The ACO metaheuristic consists of three steps that are repeated until some termination criterion is met. The three steps are solution construction, daemon actions, and pheromone updating. Daemon actions are optional. Typical termination criteria are an iteration limit or a convergence measure.

Solutions are constructed by repeatedly choosing solution components until the solution is complete. The solution components are chosen according to a probability distribution based on the pheromones. Many different probability distributions have been used by different variations of ACO. A common probability distribution is:

\begin{equation}
p(c_i)=\frac{\tau_{i}^{\alpha} \cdot \rho_{i}^{\beta}}{\sum_{c_j \in N(s^p)} \tau_{j}^{\alpha} \cdot \rho_{j}^{\beta}}
\end{equation}

Several special cases of ACO have been proposed in the literature, many predating the ACO metaheuristic. Such special cases are often tailored to specific problems and have well-defined solution construction, daemon actions, and pheromone updating. Some notable special cases are: Ant System \parencite{Dorigo1996}, Ant Colony System \parencite{Dorigo1997}, and Max-Min Ant System \parencite{Stutzle1998,Stuetzle2000}.

%end Ant Colony Optimization


\section{Labor division}
\emph{Abstract, save the details for lit. review}