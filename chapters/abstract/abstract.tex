\begin{abstract}
The project involves developing an ant colony optimization (ACO) algorithm for the minimum cost flow problem. ACO is a constructivistic and population-based metaheuristic for solving combinatorial optimization problems and has previously been applied to well-known tasks such as the traveling salesman problem. Inspired by the collective behavior of real ants in nature, the objective is to incorporate labor division in the ACO algorithm to avoid stagnation and properly balance exploration and exploitation. There is little documented research that utilizes labor division in ACO. The algorithm’s general behavior and overall performance will be tested on a substantial collection of flow networks.
\end{abstract}

\vfill
\pagebreak

\section*{Norsk sammendrag}
Intentionally blank

\begin{comment} Dette er fra forrige rapport:
Denne artikkelen evaluerer og sammenligner hvordan fire forskjellige ACO-algoritmer løser maks-flyt minimum-kost flytproblemer med konkave kostfunksjoner. Dette er et NP-hardt problem som oppstår blant annet i logistikk og økonomi. Artikkelen sammenligner de tre mest populære maur-baserte algoritmene Ant System (AS), Ant Colony System (ACS) og Max-Min Ant System (MMAS) med en ny, modifisert ACO-algoritme.

Eksperimentene utført og dokumentert i denne artikkelen beviser at ACO-algoritmer effektivt kan løse denne typen flytproblemer, men at ikke alle ACO algoritmer er like egnet. AS og ACS følger en lokal feromonoppdateringsstrategi, hvor alle veiene som alle maur har gått, får tildelt feromoner. Eksperimentene viste at dette er en dårlig strategi i denne problemdomenet, og hverken AS eller ACS klarte å løse oppgaven.

På den andre siden viste MMAS og den nye algoritmen at maur-baserte algoritmer kan løse problemet. Den nye algoritmen virket som å være en effektiv algoritme på dette domenet ettersom den både har rask konvergens og produserer løsninger med like høy kvalitet som veletablerte algoritmer i akademia.
\end{comment}